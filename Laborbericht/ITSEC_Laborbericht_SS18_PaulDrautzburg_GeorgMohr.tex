
\documentclass[
a4paper,     %% defines the paper size: a4paper (default), a5paper, letterpaper, ...
% landscape,   %% sets the orientation to landscape
% twoside,     %% changes to a two-page-layout (alternatively: oneside)
% twocolumn,   %% changes to a two-column-layout
 headsepline, %% add a horizontal line below the column title
footsepline, %% add a horizontal line above the page footer
titlepage,   %% only the titlepage (using titlepage-environment) appears on the first page (alternatively: notitlepage)
 halfparskip,     %% insert an empty line between two paragraphs (alternatively: halfparskip, ...)
% leqno,       %% equation numbers left (instead of right)
 fleqn,       %% equation left-justified (instead of centered)
% tablecaptionabove, %% captions of tables are above the tables (alternatively: tablecaptionbelow)
% draft,       %% produce only a draft version (mark lines that need manual edition and don't show graphics)
% 10pt         %% set default font size to 10 point
% 11pt         %% set default font size to 11 point
12pt         %% set default font size to 12 point
]{scrartcl}  %% article, see KOMA documentation (scrguide.dvi)



%%%%%%%%%%%%%%%%%%%%%%%%%%%%%%%%%%%%%%%%%%%%%%%%%%%%%%%%%%%%%%%%%%%%%%%%%%%%%%%%
%%%
%%% packages
%%%

%%%
%%% encoding and language set
%%%

%%% ngerman: language set to new-german
\usepackage{ngerman}

%%% babel: language set (can cause some conflicts with package ngerman)
%%%        use it only for multi-language documents or non-german ones
\usepackage[ngerman]{babel}

%%% inputenc: coding of german special characters
\usepackage[utf8]{inputenc}

%%% fontenc, ae, aecompl: coding of characters in PDF documents
\usepackage[T1]{fontenc}
\usepackage{ae,aecompl}

%%%
%%% technical packages
%%%

\usepackage{multirow}

%%% amsmath, amssymb, amstext: support for mathematics
\usepackage{amsmath,amssymb,amstext}

%%% psfrag: replace PostScript fonts
%\usepackage{psfrag}

%%% listings: include programming code
\usepackage{listings}

%%% units: technical units
\usepackage{units}

%%%
%%% layout
%%%

%%% scrpage2: KOMA heading and footer
%%% Note: if you don't use this package, please remove 
%%%       \pagestyle{scrheadings} and corresponding settings
%%%       below too.
\usepackage{scrpage2}

%%%
%%% PDF
%%%

\newif\ifpdf
  \ifx\pdfoutput\undefined
     \pdffalse
  \else
     \pdfoutput=1
     \pdftrue
  \fi

%%% Should be LAST usepackage-call!
%%% For docu on that, see reference on package ``hyperref''
\ifpdfoutput{%   (definitions for using pdflatex instead of latex)

  %%% graphicx: support for graphics
  \usepackage[pdftex]{graphicx}

  \pdfcompresslevel=9

  %%% hyperref (hyperlinks in PDF): for more options or more detailed
  %%%          explanations, see the documentation of the hyperref-package
  \usepackage[%
    %%% general options
    pdftex=true,      %% sets up hyperref for use with the pdftex program
    %plainpages=false, %% set it to false, if pdflatex complains: ``destination with same identifier already exists''
    %
    %%% extension options
    backref=true,      %% if true, adds a backlink text to the end of each item in the bibliography
    pagebackref=false, %% if true, creates backward references as a list of page numbers in the bibliography
    colorlinks=false,   %% turn on colored links (true is better for on-screen reading, false is better for printout versions)
    %
    %%% PDF-specific display options
    bookmarks=true,          %% if true, generate PDF bookmarks (requires two passes of pdflatex)
    bookmarksopen=false,     %% if true, show all PDF bookmarks expanded
    bookmarksnumbered=false, %% if true, add the section numbers to the bookmarks
    %pdfstartpage={1},        %% determines, on which page the PDF file is opened
    pdfpagemode=None         %% None, UseOutlines (=show bookmarks), UseThumbs (show thumbnails), FullScreen
  ]{hyperref}


  %%% provide all graphics (also) in this format, so you don't have
  %%% to add the file extensions to the \includegraphics-command
  %%% and/or you don't have to distinguish between generating
  %%% dvi/ps (through latex) and pdf (through pdflatex)
  \DeclareGraphicsExtensions{.pdf}

}{%else   (definitions for using latex instead of pdflatex)

  \usepackage[dvips]{graphicx}

  \DeclareGraphicsExtensions{.eps}

  \usepackage[%
    dvips,           %% sets up hyperref for use with the dvips driver
    colorlinks=false %% better for printout version; almost every hyperref-extension is eliminated by using dvips
  ]{hyperref}

}


%%% sets the PDF-Informations options
%%% (see fields in Acrobat Reader: ``File -> Document properties -> Summary'')
%%% Note: this method is better than as options of the hyperref-package (options are expanded correctly)
\hypersetup{
  pdftitle={Einrichtung einer internen Firewall von Firma-a und Firma-b und LAN-to-LAN-VPN Firma-a	(Server)	
nach	 Firma-b(Client)}, %%
  pdfauthor={}, %%
  pdfsubject={IT Sicherheitsarchitekturen}, %%
  pdfcreator={Accomplished with LaTeX2e and pdfLaTeX with hyperref-package.}, %% 
  pdfproducer={}, %%
  pdfkeywords={} %%
}


%%%%%%%%%%%%%%%%%%%%%%%%%%%%%%%%%%%%%%%%%%%%%%%%%%%%%%%%%%%%%%%%%%%%%%%%%%%%%%%%
%%%
%%% user defined commands
%%%

%%% \mygraphics{}{}{}
%% usage:   \mygraphics{width}{filename_without_extension}{caption}
%% example: \mygraphics{0.7\textwidth}{rolling_grandma}{This is my grandmother on inlinescates}
%% requires: package graphicx
%% provides: including centered pictures/graphics with a boldfaced caption below
%% 
\newcommand{\mygraphics}[3]{
  \begin{center}
    \includegraphics[width=#1, keepaspectratio=true]{#2} \\
    \textbf{#3}
  \end{center}
}

%%%%%%%%%%%%%%%%%%%%%%%%%%%%%%%%%%%%%%%%%%%%%%%%%%%%%%%%%%%%%%%%%%%%%%%%%%%%%%%%
%%%
%%% define the titlepage
%%%

 \subject{IT Sicherachitekturen}   %% subject which appears above titlehead
% \titlehead{} %% special heading for the titlepage

%%% title
\title{Einrichtung einer internen Firewall von Firma-a und Firma-b, sowie LAN-to-LAN-VPN von Firma-a (Server)	
nach	 Firma-b(Client)}

%%% author(s)
\author{Paul Drautzburg \and
Georg Mohr}

%%% date
\date{HTWG Konstanz, Sommersemester 2018}

% \publishers{}

% \thanks{} %% use it instead of footnotes (only on titlepage)

% \dedication{} %% generates a dedication-page after titlepage


%%% uncomment following lines, if you want to:
%%% reuse the maketitle-entries for hyperref-setup
%\newcommand\org@maketitle{}
%\let\org@maketitle\maketitle
%\def\maketitle{%
%  \hypersetup{
%    pdftitle={\@title},
%    pdfauthor={\@author}
%    pdfsubject={\@subject}
%  }%
%  \org@maketitle
%}


%%%%%%%%%%%%%%%%%%%%%%%%%%%%%%%%%%%%%%%%%%%%%%%%%%%%%%%%%%%%%%%%%%%%%%%%%%%%%%%%
%%%
%%% set heading and footer
%%%

%%% scrheadings default: 
%%%      footer - middle: page number
\pagestyle{scrheadings}

%%% user specific
%%% usage:
%%% \position[heading/footer for the titlepage]{heading/footer for the rest of the document}

%%% heading - left
% \ihead[]{}

%%% heading - center
% \chead[]{}

%%% heading - right
 %\ohead[]{}

%%% footer - left
% \ifoot[]{}

%%% footer - center
% \cfoot[]{}

%%% footer - right
% \ofoot[]{}



%%%%%%%%%%%%%%%%%%%%%%%%%%%%%%%%%%%%%%%%%%%%%%%%%%%%%%%%%%%%%%%%%%%%%%%%%%%%%%%%
%%%
%%% begin document
%%%

\begin{document}

 \pagenumbering{roman} %% small roman page numbers

%%% include the title
% \thispagestyle{empty}  %% no header/footer (only) on this page
 \maketitle

%%% start a new page and display the table of contents
 \newpage
 \tableofcontents

%%% start a new page and display the list of figures
 \newpage
 \listoffigures

%%% start a new page and display the list of tables
\newpage
 \listoftables

%%% display the main document on a new page 
 \newpage

 \pagenumbering{arabic} %% normal page numbers (include it, if roman was used above)

%%%%%%%%%%%%%%%%%%%%%%%%%%%%%%%%%%%%%%%%%%%%%%%%%%%%%%%%%%%%%%%%%%%%%%%%%%%%%%%%
%%%
%%% begin main document
%%% structure: \section \subsection \subsubsection \paragraph \subparagraph
%%%

\section{Motivation}
Im Zeitalter der Digitalisierung steigt der Grad der Vernetzung immer weiter an. Unternehmen, welche ihre Standorte über die ganze Welt verteilt haben, wollen Wege und Möglichkeiten haben ihre Arbeit sicher und Problemlos zu verrichten. Dafür ist es unerlässlich, dass verschiedene Standorte auf Ressourcen des anderen zugreifen können. Für solche Szenarien gibt es in der Netzwerktechnik und Netzwerkarchitektur verschiedene Lösungsansätze. 
Jedoch darf dabei ein wichtiger Aspekt nicht vernachlässigt werden, denn jedes Unternehmen hat Betriebsmittel und Informationen, welche es nicht Gefahren von außen aussetzten will. An diesem Punkt kommt der Begriff IT-Sicherheit ins Spiel und genau dieses Umfeld um den Begriff „IT-Sicherheit“ wird im Rahmen des Praktikums zur Lehrveranstaltung „IT-Sicherheitsarchitekturen“ untersucht. 
In den folgenden Kapiteln wird eine Problemstellung unter Laborbedingungen skizziert, welche ein Szenario wiederspiegelt, mit dem sich Unternehmen täglich konfrontiert sehen. Anschließend wird werden die Laborbedienungen und der Versuchsaufbau beschrieben. Aus der Problemstellung und dem Versuchsaufbau wird eine Lösung auf Grundlage, der damit zu Grunde liegenden Theorie, erarbeitet. Diese Lösung wird abschließend in die Laborumgebung implementiert und auf ihre Akzeptanz gegenüber der Problemstellung untersucht und in einem Fazit bewertet.   
\section{Problemstellung}
In diesem Kapitel wird die Laborumgebung eingeführt und beschrieben, sowie die damit verbundene Aufgaben-/Problemstellung skizziert.
\subsection{Laborumgebung und Versuchsaufbau}
\label{laborumgebung}
Im Labor wird die IT-Landschaft zweier Unternehmen „A“ und „B“ simuliert. Jedes Unternehmen besitzt ein lokales LAN, an denen die pot. Mitarbeiter angeschlossen sind, zudem gibt es mehrere Server mit Hilfe derer verschiedene Dienste angeboten werden sollen. 
Es folgt eine Liste an Diensten welche Angeboten werden sollen:

\begin{itemize}
\item Internetzugang von den Mitarbeiter-Computern.
\item Zugang über HTTPS der Webseiten beider Unternehmen.
\item Eine Virtuelle Kopplung der beiden Unternehmen mit Hilfe einer sog. Site-to-Site Verbindung.
\item Jeder Mitarbeiter der Unternehmen soll die Möglichkeit haben, E-Mails über einen sicheren  Mailserver zu verschicken.  
\end{itemize}
Eine weitere wichtige Komponente in dieser IT-Landschaft sind die externen und internen Firewalls, da diese den Grundstein für eine sichere Umgebung legen. In dieser IT-Landschaft gibt es insgesamt 4 Firewalls, jedes Unternehmen hat jeweils eine interne und externe Firewall. Hinter den externen Firewalls beider Unternehmen liegt jeweils die sog. DMZ – Demilitarisierte Zone. In dieser Zone stehen üblicherweise die Unternehmensserver, diese werden nach außen hin von der externen Firewall und nach innen von der internen Firewall geschützt. Jede Firewall läuft auf einem Server, somit besteht die IT-Landschaft insgesamt aus 4 Servern auf welchen die Firewalls laufen, 2 Servern, welche die jeweiligen Unternehmensserver simulieren und einem Server welcher außerhalb der externen Firewall das Internet simulieren soll (vgl. Abb. xxx).

Internetzugang von den Mitarbeiter-Computern.
Zugang über HTTPS der Webseiten beider Unternehmen.
Eine Virtuelle Kopplung der beiden Unternehmen mit Hilfe einer sog. Site-to-Site Verbindung.
Jeder Mitarbeiter der Unternehmen soll die Möglichkeit haben, E-Mails über einen Mailsserver zu verschicken.  
Eine weitere wichtige Komponente in dieser IT-Landschaft sind die externen und internen Firewalls, da diese den Grundstein für eine sichere Umgebung legen. In dieser IT-Landschaft gibt es insgesamt 4 Firewalls, jedes Unternehmen hat jeweils eine interne und externe Firewall. Hinter den externen Firewalls beider Unternehmen liegt jeweils die sog. DMZ – Demilitarisierte Zone. In dieser Zone stehen üblicherweise die Unternehmensserver, diese werden nach außen hin von der externen Firewall und nach innen von der internen Firewall geschützt. Jede Firewall läuft auf einem Server, somit besteht die IT-Landschaft insgesamt aus 4 Servern auf welchen die Firewalls laufen, 2 Servern, welche die jeweiligen Unternehmensserver simulieren und einem Server welcher außerhalb der externen Firewall das Internet simulieren soll (vgl. Abb. \ref{fig:appStat}).

\begin{figure}[h]
	\includegraphics[width=\textwidth]{pictures/laborumgebung.png}
	\caption{Schematischer Aufbau der Laborumgebung im IT-Sicherheitslabor \cite{JueNeuSaDue}}
	\label{fig:appStat}
\end{figure}

Jede dieser angesprochenen Komponenten muss entsprechen konfiguriert werden, damit die angesprochenen Dienste fehlerfrei funktionieren können.
Technisch gesehen wird jeder dieser Server auf Grundlage einer virtuellen Maschine abgebildet.
Jede dieser Maschinen hat Linux als Betriebssystem installiert und vorkonfiguriert, jedoch ohne zusätzliche Pakte, nur die reine Grundinstallation   
Da die Implementierung und Installation aller dieser Dienste und der damit verbundenen Komponenten den Rahmen dieses Praktikums sprengen würde, werden die Aufgabenstellungen zuvor runter gebrochen. Die detaillierte Aufgabenstellung wird im nächsten Abschnitt eingeführt.

\subsection{Aufgabenstellung}\label{kap:Aufgabenstellung}
Wie schon im vorhergehenden Abschnitt beschrieben, gibt es mehrere Dienste, welche in der IT-Landschaft angeboten werden sollen. 
In diesem Bericht, wird die Virtuelle Kopplung der Unternehmen „A“ und „B“ durch eine „Site-to-Site-“ oder auch „LAN-to-LAN-“ genannt -Verbindung implementiert. 
Die Problemstellung gilt als gelöst, wenn es möglich ist, dass Unternehmen A eine gesicherte Verbindung zu Unternehmen B aufbauen kann und die jeweiligen internen Firewalls entsprechend konfiguriert sind. Die Konfiguration der externen Firewalls kann in dieser Aufgabe vernachlässigt werden, hierfür sollen die bereitgestellten internen Firewalls verwendet werden. 
//TODO Verweis auf Bilder, Begriffe zuordnen. 
 

\section{Theoretische Grundlagen}
In diesem Kapitel werden die theoretischen Grundlagen eingeführt, welche im Kontext der Aufgabestellung benötigt werden. Dies beschränkt sich auf drei Bereiche, die Netzwerkarchitekturen, Debian GNU Linux 7.11 und IP-Tables. Jede dieser genannten Bereiche wird innerhalb dieses Kapitels eingeführt und in den Aufgabenkontext eingeordnet. 
\subsection{Netzwerkarchitekturen}
Der Bereich Netzwerkarchitekturen ist ein sehr breites Gebiet, deshalb muss dieser zuvor anhand der Anforderungen der Aufgabenstellung abgegrenzt werden.
\subsubsection{Die Begriffe Intranet, DMZ und Internet}
Das Netzwerk des Labors besteht netzwerktechnisch aus drei Bereichen dem Intranet, der Demilitarisierten Zone und dem Internet. 
Diese Begriffe werden folgend für den Rahmen dieses Berichts definiert und in den Kontext der Problemstellung eingeordnet.
\begin{quotation}
\item["Intranet"]

Nach dem Gabler Wirtschaftslexikon wird ein Intranet als " ein unternehmensinternes Kommunikationsnetz, in dem Daten auf der Basis der Protokollfamilie TCP/IP übertragen werden" definiert. In dieser Arbeit wird der Aspekt der Protokollfamilie "TCP/IP" von vorrangiger Bedeutung sein. Der Punkt des "unternehmensinternen Kommunikationsnetzes", kann für diese Arbeit in den Hintergrund gestellt werden, da sich das Laborpraktikum auf die technischen Umsetzung beschränkt. 
Ein weiterer wichtiger Aspekt des Intranets ist, dass es nur autorisierten Benutzern gestattet werden soll zuzugreifen.  

\item["Demilitarisierte Zone"]

Der Begriff demilitarisierte Zone kommt ursprünglich aus dem militärischen Umfeld und beschreibt eine Zone oder Bereich in der sich keine militärischen Streitkräfte gegenüberstehen dürfen, quasi ein neutraler Bereich. 
In der Netzwerktechnik wird dieser Begriff benutzt, um eine Zone zu beschreiben welche sich zwischen zwei Schutzeinrichtungen befindet. Bei diesen Schutzeinrichtungen handelt es sind meistens um eine externe Firewall und eine interne Firewall. Der Hintergrund für die Einrichtung einer DMZ ist es die Sicherheit der Komponenten und Teilnehmer eines Intranets zu sichern, falls es einen Angriff auf die Komponenten innerhalb der DMZ gibt und diese korumpiert werden sollten. Die Komponenten innerhalb einer DMZ werden als potentielle Opfer oder "Victims" bezeichnet. Es handelt sich hier meistens um Server welche nach einem Schadensfall einfach durch einen Reboot wiederhergestellt werden können. 
Als Alternative könnte statt der DMZ ein sog. Application Layer Gateway (APL) eingesetzt werden. Dieser zeichnet sich dadurch, dass er den Netzwerkverkehr komplett auftrennt und sich nach allen Seiten als Kommunikationspartner verhält. Zudem wäre auch eine Lösung aus DMZ und APL\cite{intra1}. Im Rahmen des Praktikums wird jedoch lediglich eine DMZ eingesetzt.

\item["Internet"]
Das Internet ist "ein weltumspannendes, heterogenes Computernetzwerk, das auf dem Netzwerkprotokoll TCP/IP basiert. Über das Internet werden zahlreiche Dienste wie z.B. E-Mail, FTP, World Wide Web (WWW) oder IRC angeboten" \cite{Gab-Internet}. Ein wichtiger Punkt im Rahmen dieses Praktikums ist, dass das Internet innerhalb der Laborumgebung nur simuliert wird, es besteht also kein richtiger Zugang zum Internet. Dieser Fakt stellt sicher, dass der Versuchsaufbau bei falscher Konfiguration, von außen beschädigt werden könnte. Zudem wird somit sichergestellt, dass bei der Konfiguration der virtuellen Maschinen nur die Versionen der zu Grunde liegenden Images verwendet werden können. 

\end{quotation}

Nachdem die Begriffe Intranet, DMZ und Internet im Kontext der Laborumgebung und Problemstellung definiert und eingrenzt wurden müssen noch weitere Begriffe eingeführt werden. Im folgenden Abschnitt werden die Netzwerkprotokolle oder auch Kommunikationsprotokolle eingeführt, welche im Rahmen des Praktikums verwendet werden. 

\subsection{Netzwerkprotokolle/Kommunikationsprotokolle} 
Um eine Netzwerkkommunikation verschiedener Komponenten innerhalb eines Netzwerks zu gewährleisten müssen Netzwerkprotokolle eingesetzt werden. 

Im Rahmen des Laborpraktikums werden verschiedene Netzwerkprotokolle eingesetzt, diese lassen sich am besten der entsprechenden Layern im OSI-Schichtenmodell beschrieben. 
In der nachfolgenden Tabelle werden anhand der Schichten die einzelnen Protokolle eingeordnet, anschließend werden die für dieses Praktikum relevanten Protokolle gesondert beschrieben. 

\begin{table}[!h]
\centering
\resizebox{\textwidth}{!}{%
\begin{tabular}{|c|l|l|l|l|}
\hline
\textbf{\#} & \multicolumn{1}{c|}{\textbf{OSI-Schicht}} & \multicolumn{1}{c|}{\textbf{Einordnung}} & \multicolumn{1}{c|}{\textbf{Protokolle}} & \multicolumn{1}{c|}{\textbf{Kopplungselemente}} \\ \hline
\textbf{7} & Anwendungen(Application) & \multirow{3}{*}{\begin{tabular}[c]{@{}l@{}}Anwendungs-\\ orientiert\end{tabular}} & \multirow{3}{*}{HTTP, FTP, HTTPS, SMTP, DNS, LDAP} & \multirow{4}{*}{\begin{tabular}[c]{@{}l@{}}Gateway\\ Content-Switch \\ Proxy\\ Layer-4-7-Switch\end{tabular}} \\ \cline{1-2}
\textbf{6} & Darstellung(Presentation) &  &  &  \\ \cline{1-2}
\textbf{5} & Sitzung(Session) &  &  &  \\ \cline{1-4}
\textbf{4} & Transport(Transport) & \multirow{4}{*}{\begin{tabular}[c]{@{}l@{}}Transport-\\ orientiert\end{tabular}} & TCP, UDP, SCTP, SPX &  \\ \cline{1-2} \cline{4-5} 
\textbf{3} & Vermittlung-/Paket(Network) &  & ICMP, IGMP, IP, IPsec, IPX & RouterLayer-3-Switch \\ \cline{1-2} \cline{4-5} 
\textbf{2} & Sicherung(Data Link) &  & \multirow{2}{*}{Ethernet, Token Ring, FDDI} & BridgeLayer-2-Switch \\ \cline{1-2} \cline{5-5} 
\textbf{1} & Bitübertragung(Physical) &  &  & \begin{tabular}[c]{@{}l@{}}Netzwerkkabel\\ Repeater\\ Hub\end{tabular} \\ \hline
\end{tabular}%
}
\caption{OSI-Schichtenmodell}
\label{OSI}
\end{table}

Die bereits beschriebene Aufgabenstellung (vgl. \ref{kap:Aufgabenstellung} \nameref{kap:Aufgabenstellung}) soll eine site-to-site VPN eingerichtet werden. Da es um eine die Verbindung geht, kann der Fokus auf die Transport-orientierten Schichten der OSI-Modells 1-4 gerichtet werden. Für die Umsetzung einer VPN-Verbindung und Konfiguration der internen Firewalls sind die folgenden Protokolle vordergründig
\begin{quotation}
\item ["ICMP"] Das Internet Control Message Protocol, wird dazu verwendet in Rechnernetzen einen Austausch von Fehlermeldungen durchzuführen. 
\item ["IP"] Das Internet Protocol ist ein zustands- und verbindungsloses Protokoll, welches die Implementierung der Vermittlungsschicht (3) widerspiegelt. 
\item ["IPsec"] Das Internet Protocol Security ist eine Erweiterung des IP-Protokolls und soll eine gesicherte Kommunikation über unsichere IP-Netze ermöglichen.  
\item ["TCP"] Das Transmission Control Protocol wird von allen modernen Betriebssystemen genutzt um, zu definieren, wie die Daten zwischen verschiedenen Netzwerkkomponenten ausgetauscht werden sollen 
\item ["UDP"] Das User Datagram Protocol zeichnet sich dadurch aus, dass es verbindungslos und ungeschützt ist. Es kann also nicht gesichert werden, ob ein gesendetes Datenpaket richtig (nicht verfälscht von Dritten) oder überhaupt  ankommt. \cite{Kurose2008Computernetzwerke}
\end{quotation}
Diese beschriebenen Protokolle sind der Lösung der Aufgabe von großer Bedeutung. Sie werden bei der Erstellung der Lösungsskizze wieder aufgegriffen. Um die Beschreibung Netzwerkprotokolle zu komplettieren muss noch auf eine Ebene tiefer geschaut werden, auf die Datenpakte. Im nächsten Abschnitt wird der Aufbau und die wichtigsten Eigenschaften eines Datenpakets beschrieben. 

\subsubsection{Aufbau eins Datenpakets}
Im vorhergehenden Abschnitt wurden die wichtigsten Netzwerkprotokolle für den Rahmen des Laborpraktikums beschrieben. In diesem Abschnitt, wird das Zusammenspiel eines Netzwerksprotokoll und Datenpaketen gezeigt. 
 
In einem Protokoll wird der Aufbau eines Datenpakets definiert, zudem enthält es wichtige Informationen über den Datenaustausch. 
Es definiert, 
\begin{itemize}
\item wer der Absender und Empfänger eines Datenpaktes sein soll,
\item von welchem Typ ein Datenpaket ist, ob es für den Verbindungsaufbau, Verbindungsabbau oder für Nutzdaten genutzt wird, 
\item die Größe des Datenpaket, welches beim Empfänger ankommen soll. 
\end{itemize}
Wenn es sich um eine mehrteilige Kommunikation handelt, muss zusätzlich noch die laufende Nummer und die Anzahl der Pakte definiert werden. Zuletzt folgt noch eine Prüfsumme, mit dessen Hilfe der Empfänger prüfen kann, ob die Datenpakete fehlerfrei angekommen sind. 
Alle dieser beschriebenen Informationen werden dem "Header" eines Datenpaktes vorne oder hinten angehängt, dieser angehängte Bereich wird auch "Trailer genannt". Anhand der folgenden Abbildung \ref{fig:datenpaket} wird der Aufbau eines Datenpakts und der einzelnen Bestandteile noch verdeutlicht. 
\begin{figure}[!h]
	\includegraphics[width=\textwidth]{pictures/datenpaket.png}
	\caption{Aufbau eines Datenpakets \cite{Header-Dat}}
	\label{fig:datenpaket}
\end{figure}

\subsection{Debian GNU Linux 7.11}
\subsection{IP-Tables}\label{iptables}
\subsection{OpenVPN}
\subsection{Public Key Intrastructure}
% !TEX encoding = UTF-8 Unicode
\section{Lösungsskizze}
In diesem Kapitel wird sich mit dem Thema befasst, wie sich eine Side-to-Side VPN Verbindung, die die Firmen A und B wie in Kapitel \ref{laborumgebung} beschrieben verbindet.
Dazu waren zu Beginn mehrere Vorüberlegungen zu tätigen. Dies sind zum einen, wie wird das innere Netz der Firmen A und B geschützt, dann welcher VPN Dienst kann für diesen Zweck verwendet werden und welche weiteren Strategien braucht es bei der Umsetzung der Security Policies der Firmen. 
\subsection{Firewall}
Wenn eine Firewall aufgebaut werden soll, muss sich zu beginn überlegt werden, welche allgemeine Strategie mit der Firewall gefahren werden soll. Das heißt sollen alle Verbindungen Standardmäßig freigegeben werden und nur explizit nicht erlaubt Verbindungen geblockt werden (Default-Allow-Strategy), oder sollen alle Verbindungen blockiert werden und nur die die explizit erlaubt wurden, geöffnet werden (Default-Deny-Strategy). Im Standardfall wird in Unternehmen, die Strategie Default-Deny verwendet, weshalb auch im Versuchsaufbau diese Strategie gewählt wurde. Eine Firewall wird mit Hilfe des Userspace-Programmes IPTABLES (siehe Kapitel \ref{iptables}) unter Linux realisiert. Dafür wird eine Rules Datei (nachfolgend Skript genannt) angelegt, in dieser werden die Einstellungen und Regeln für die Firewall definiert. Zu Beginn des Skriptes wird der Interpreter für den Code angegeben. Dann wird begonnen die durch die Default Strategie vorgegebenen Default Policies umzusetzen (siehe Auflistung \ref{lst:defaultPolicy}). Dabei sagt das -P, dass die Policy angesprochen werden soll, das INPUT, OUTPUT und FORWARD bezieht sich auf die im Kapitel \ref{iptables} vorgestellten Chains. DROP sagt dabei aus, dass die eintreffenden Pakete, wenn keine weiteren Regeln definiert oder zutreffen nicht weitergeleitet und ``fallengelassen'' werden sollen.
\newline
\lstset{
	basicstyle=\footnotesize, frame=tb,
	xleftmargin=.2\textwidth, xrightmargin=.2\textwidth
}
\begin{lstlisting}[caption={Default Policy IPTABLE},label=lst:defaultPolicy]
#Verwenden der Tabelle FILTER
*filter
-P INPUT DROP
-P OUTPUT DROP
-P FORWARD DROP
\end{lstlisting}
\vspace{\baselineskip}
Nach dem die Default Policy erfolgreich eingeführt wurde, werden die feingranulareren Regeln definiert. Dies setzt die Verwendung des Befehles -F voraus. Dieser löscht alle bisherigen Filterregeln, um die nach folgenden neu definierten Regeln einzuführen. Bei Linux ist dabei zu beachten das ein Teil der Interprozesskommunikation über das interne Netzwerk läuft. Dafür ist es nötig diese Kommunikation zuzulassen, dies geschieht wie in der nachfolgenden Auflistung \ref{lst:interprozess} zusehen ist. 
\newline
\lstset{
	basicstyle=\footnotesize, frame=tb,
	xleftmargin=.2\textwidth, xrightmargin=.2\textwidth
}
\begin{lstlisting}[caption={Interprozesskommunikation zulassen},label=lst:interprozess]
# Interprozesskommunikation Verbindungen erlauben
-A INPUT -i lo -j ACCEPT
-A OUTPUT -o lo -j ACCEPT
\end{lstlisting}
\vspace{\baselineskip}
Dabei sagt das -A an welche Chain diese Regel angehängt werden soll. Das -i ist dabei die Option über welches Netzwerkinterface das Paket eingegangen ist, beziehungsweise -o für das Paket versenden. In diesem Fall das ``lo'' Netzwerkinterface. Wenn alle Prüfregeln auf das Paket zutreffen wird mit -j entschieden wie mit dem Paket verfahren werden soll, in diesem Beispiel soll es mit ACCEPT akzeptiet werden. 
\newline
\lstset{
	basicstyle=\footnotesize, frame=tb,
	xleftmargin=.2\textwidth, xrightmargin=.2\textwidth
}
\begin{lstlisting}[caption={Weitere allgemeine Firewallregeln},label=lst:allgemeineRegeln]
# Erlaube ICMP Befehle
-A INPUT -p icmp -j ACCEPT
-A OUTPUT -p icmp -j ACCEPT

# Erlaube SSH Verbindung
-A INPUT -p tcp --dport 22 -j ACCEPT

# Alle Verbindungen von innen nach aussen zulassen
-A FORWARD -i eth0 -o eth1 -m state --state NEW -j ACCEPT

# Erlaube nur bereits aufgebaute 
# Verbindungen von aussen nach innen
-A FORWARD -m state --state RELATED,ESTABLISHED -j ACCEPT
\end{lstlisting}
\vspace{\baselineskip}
Die in Auflistung \ref{lst:allgemeineRegeln} dargestellten Regeln sind allgemeine Regeln für die Firewall, diese erlauben das empfangen von Pinganfragen, den Fernzugriff mittels SSH, alle neuen Verbindungen von innen nach außen und alle bereits aufgebauten Verbindungen beziehungsweiße alle Verbindungen die einen Bezug auf eine andere Verbindung besitzen. In der nachfolgenden Tabelle \ref*{tab:iptablesOptionen} werden die hier verwendeten Optionen nochmals näher beschrieben.
\begin{table}[h]
	\centering
\begin{tabular}{|p{2cm}|p{14cm}|}
	\hline 
	Optionen & Beschreibung \\ 
	\hline 
	-p & Gibt das Protokoll an welches verwendet werden soll hier icmp(Ping), tcp \\ 
	\hline 
	-dport & Gibt den Destinationport an, auf den zugegriffen werden soll hier 22 \\ 
	\hline
	eth0,eth1 & bezieht sich auf das verwendete Netzwerkinterface \\ 
	\hline 
	-m state & die entreffenden Pakete sollen auf den Status überprüft werden \\ 
	\hline 
	--state & Status der eintreffenden Pakete, hier NEW, RELATED,ESTABLISHED \\ 
	\hline 
\end{tabular} 
\caption{IPTABLES Optionen und Beschreibung} \label{tab:iptablesOptionen}
\end{table}
Nach den allgemeine Regeln müssen nun die VPN spezifischen Regeln definiert werden. Dazu muss man die auf den VPN Ports eintreffenden und ausgehenden Pakete betrachten. Diese Ports sind entweder 1194 oder 1195. In der Auflistung \ref{lst:VPNRegeln} sind VPN Regeln zusehen. Eine Besonderheit bei VPN ist das zum einen alle verbinungen die über das Netzwerkinterface tun0 eintreffenden Pakete ohne Überprüfung an das Inteface eth0 übergeben werden. Des weiteren wird noch die Tabelle NAT benötigt, was das ''*nat'' angibt. In dieser Tabelle gibt es die Chain POSTROUTING, über diese kann nachträglich der Verkehrsheader eines Paketes verändert werden. MASQUERADE hat dabei die Funktion das wenn ein Paket versendet wird die Source-IP-Adresse so verändert wird das nur noch die IP-Adresse des Firewallservers ersichtlich ist. Dies hat den Grund das von außen nicht ersichtlich wird, was sich für IP-Adressen hinter der Firewall befinden und es so Angreifern erschwert wird diese anzugreifen.
\lstset{
	basicstyle=\footnotesize, frame=tb,
	xleftmargin=.2\textwidth, xrightmargin=.2\textwidth
}
\begin{lstlisting}[caption={Weitere VPN Firewallregeln},label=lst:VPNRegeln]
# Erlaube alle Verbindungen auf den VPN Ports
-A INPUT -i eth1 -p udp --dport 1194 -m state --state NEW -j ACCEPT
-A INPUT -i eth1 -p udp --dport 1195 -m state --state NEW -j ACCEPT
-A INPUT -m state --state RELATED,ESTABLISHED -j ACCEPT
-A OUTPUT -m state --state RELATED,ESTABLISHED -j ACCEPT
-A INPUT -i tun0 -j ACCEPT

#Forwarding fuer die VPN Verbindungen
-A FORWARD -i tun0 -o eth0 -m state --state NEW -j ACCEPT
-A FORWARD -i eth0 -o tun0 -m state --state NEW -j ACCEPT
-A FORWARD -i tun0 -o eth0 -m state --state RELATED,ESTABLISHED -j ACCEPT
-A FORWARD -i eth0 -o tun0 -m state --state RELATED,ESTABLISHED -j ACCEPT
COMMIT

#Verwenden der Tabelle NAT
*nat

#Loeschen der vorhandenen Regeln
-F

-A POSTROUTING -o eth1 -j MASQUERADE
COMMIT
\end{lstlisting}

\subsection{VPN}
\subsubsection{Server Firma A}
\subsubsection{Client Firma B}











% !TEX encoding = UTF-8 Unicode
\section{Auswertung}
In diesem Kapitel werden die im Kapitel \ref{loesungsskizze} vorgestellten Konfigurationen verwendet und eine Auswertung der VPN Verbindung von Firma B(Client) zur Firma A(Server) gemacht.\newline
Um die VPN Verbindung zu starten wird zuerst der Server mit Hilfe des Befehls ''openvpn /etc/openvpn/server.conf'' gestartet. Nach dem der Server gestartet wurde, wird bei Firma B der Client mit ''openvpn /etc/openvpn/client.conf'' gestartet. Dabei initiiert der Client den Verbindungsaufbau. 
\begin{figure}[h]
	\includegraphics[width=\textwidth]{pictures/vpnVerbindungsaufbau.png}
	\caption{Verbindungsaufbau eines VPN's\cite{vanRijn2018May}}
	\label{fig:SSlhand}
\end{figure}
Die Abbildung \ref{fig:SSlhand} stellt drei Wege SSL Handshake zwischen Client und Server dar. Dabei wird zu Beginn eine unverschlüsselte Verbindung aufgebaut, indem dem Client die Server-IP und der Port mitgeteilt wird. Nachdem diese Verbindung ''ESTABLISHED'' ist, sendet der Client einen ''Client Hallo'' auf die nun bekannte IP-Adresse und den Port. Dabei liefert er das von ihm verwendete SSL Protokoll, eine Session ID, eine Liste von Chiffren, die er verwenden kann an den Server aus. Auf dieses ''Hallo'' sendet der Server selbst ein ''Server Hallo'' und liefert dabei dem Client ebenfalls seine SSL Protokoll Version, die Session ID, das von ihm gewählten Chiffre, sein Serverzertifikat und gegebenenfalls seine CCR (Client Certificate Request) aus. Hat der Client den ''Server Hallo'' erhalten, überprüft dieser die Gültigkeit, das Vertrauen in die CA und den öffentlichen Schlüssel mit der digitalen Signatur. 

Wenn er dies überprüft hat, erzeugt der Client, mit Hilfe des öffentlichen Schlüssels des Servers, ein Pre-master-secret. Dieses Secret sendet er nun an den Server. Der Server entschlüsselt das Pre-master-secret und anschließend erstellen Server und Client aus dem pre-master-secret einen symmetrischen Schlüssel, über den die Kommunikation verschlüsselt und entschlüsselt werden kann.\newline
Im Labor stießen wir auf ein nicht zu überbrückendes Hindernis. Dies entstand nach dem TCP-Handshake, im SSL-Handshake. Dabei konnte der Server sein ''Server Hallo'' nicht an den Client zurückschicken, was zur Folge hatte, dass der Client kein pre-master-secret erstellen konnte und der Vorgang abgebrochen wurde. Dieser Vorgang wurde nach 120 Sekunden erneut angestoßen und es wurde ein erneuter Verbindungsversuch unternommen, jedoch mit dem gleichen Resultat. Dieser Vorgang wird durch die Abbildung \ref{fig:logClient} und Abbildung \ref{fig:logServer} deutlich. 
\begin{figure}[h]
	\includegraphics[width=\textwidth]{pictures/clientlog.png}
	\caption{Log Ausgabe des gestarteten VPN-Client}
	\label{fig:logClient}
\end{figure}
\begin{figure}[h]
	\includegraphics[width=\textwidth]{pictures/Serverlog.png}
	\caption{Log Ausgabe des gestarteten VPN-Servers}
	\label{fig:logServer}
\end{figure}
Nach Stunden der Fehlersuche und Neukonfiguration auf verschiedenen Systemen trat der Fehler erneut auf. Dies erhärtet die Vermutung, dass der Fehler in den Konfigurationen der virtuellen Maschinen zu finden sein könnte. Diese Vermutung konnte allerdings nicht bestätigt werden.  
\section{Fazit}


%%%
%%% end main document
%%%
%%%%%%%%%%%%%%%%%%%%%%%%%%%%%%%%%%%%%%%%%%%%%%%%%%%%%%%%%%%%%%%%%%%%%%%%%%%%%%%%

 \appendix  %% include it, if something (bibliography, index, ...) follows below

%%%%%%%%%%%%%%%%%%%%%%%%%%%%%%%%%%%%%%%%%%%%%%%%%%%%%%%%%%%%%%%%%%%%%%%%%%%%%%%%
%%%
%%% bibliography
%%%
%%% available styles: abbrv, acm, alpha, apalike, ieeetr, plain, siam, unsrt
%%%
 \bibliographystyle{alphadin}

%%% name of the bibliography file
 \bibliography{bib/literatur.bib}

\end{document}
%%% }}}
%%% END OF FILE
%%%%%%%%%%%%%%%%%%%%%%%%%%%%%%%%%%%%%%%%%%%%%%%%%%%%%%%%%%%%%%%%%%%%%%%%%%%%%%%%
%% Local Variables:
%% mode: outline-minor
%% OPToutline-regexp: "%% .*"
%% OPTeval: (hide-body)
%% emerge-set-combine-versions-template: "%a\n%b\n"
%% End:
%% vim:foldmethod=marker
