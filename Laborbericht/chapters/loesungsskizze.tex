% !TEX encoding = UTF-8 Unicode
\section{Lösungsskizze}
In diesem Kapitel wird sich mit dem Thema befasst, wie sich eine Side-to-Side VPN Verbindung, die die Firmen A und B wie in Kapitel \ref{laborumgebung} beschrieben verbindet.
Dazu waren zu Beginn mehrere Vorüberlegungen zu tätigen. Dies sind zum einen, wie wird das innere Netz der Firmen A und B geschützt, dann welcher VPN Dienst kann für diesen Zweck verwendet werden und welche weiteren Strategien braucht es bei der Umsetzung der Security Policies der Firmen. 
\subsection{Firewall}\label{firewallloesung}
Wenn eine Firewall aufgebaut werden soll, muss sich zu beginn überlegt werden, welche allgemeine Strategie mit der Firewall gefahren werden soll. Das heißt sollen alle Verbindungen Standardmäßig freigegeben werden und nur explizit nicht erlaubt Verbindungen geblockt werden (Default-Allow-Strategy), oder sollen alle Verbindungen blockiert werden und nur die die explizit erlaubt wurden, geöffnet werden (Default-Deny-Strategy). Im Standardfall wird in Unternehmen, die Strategie Default-Deny verwendet, weshalb auch im Versuchsaufbau diese Strategie gewählt wurde. Eine Firewall wird mit Hilfe des Userspace-Programmes IPTABLES (siehe Kapitel \ref{iptables}) unter Linux realisiert. Dafür wird eine Rules Datei (nachfolgend Skript genannt) angelegt, in dieser werden die Einstellungen und Regeln für die Firewall definiert. Zu Beginn des Skriptes wird der Interpreter für den Code angegeben. Dann wird begonnen die durch die Default Strategie vorgegebenen Default Policies umzusetzen (siehe Auflistung \ref{lst:defaultPolicy}). Dabei sagt das -P, dass die Policy angesprochen werden soll, das INPUT, OUTPUT und FORWARD bezieht sich auf die im Kapitel \ref{iptables} vorgestellten Chains. DROP sagt dabei aus, dass die eintreffenden Pakete, wenn keine weiteren Regeln definiert oder zutreffen nicht weitergeleitet und ``fallengelassen'' werden sollen.
\newline
\lstset{
	basicstyle=\footnotesize, frame=tb,
	xleftmargin=.2\textwidth, xrightmargin=.2\textwidth
}
\begin{lstlisting}[caption={Default Policy IPTABLE},label=lst:defaultPolicy]
#Verwenden der Tabelle FILTER
*filter
-P INPUT DROP
-P OUTPUT DROP
-P FORWARD DROP
\end{lstlisting}
\vspace{\baselineskip}
Nach dem die Default Policy erfolgreich eingeführt wurde, werden die feingranulareren Regeln definiert. Dies setzt die Verwendung des Befehles -F voraus. Dieser löscht alle bisherigen Filterregeln, um die nach folgenden neu definierten Regeln einzuführen. Bei Linux ist dabei zu beachten das ein Teil der Interprozesskommunikation über das interne Netzwerk läuft. Dafür ist es nötig diese Kommunikation zuzulassen, dies geschieht wie in der nachfolgenden Auflistung \ref{lst:interprozess} zusehen ist. 
\newline
\lstset{
	basicstyle=\footnotesize, frame=tb,
	xleftmargin=.2\textwidth, xrightmargin=.2\textwidth
}
\begin{lstlisting}[caption={Interprozesskommunikation zulassen},label=lst:interprozess]
# Interprozesskommunikation Verbindungen erlauben
-A INPUT -i lo -j ACCEPT
-A OUTPUT -o lo -j ACCEPT
\end{lstlisting}
\vspace{\baselineskip}
Dabei sagt das -A an welche Chain diese Regel angehängt werden soll. Das -i ist dabei die Option über welches Netzwerkinterface das Paket eingegangen ist, beziehungsweise -o für das Paket versenden. In diesem Fall das ``lo'' Netzwerkinterface. Wenn alle Prüfregeln auf das Paket zutreffen wird mit -j entschieden wie mit dem Paket verfahren werden soll, in diesem Beispiel soll es mit ACCEPT akzeptiet werden. 
\newline
\lstset{
	basicstyle=\footnotesize, frame=tb,
	xleftmargin=.2\textwidth, xrightmargin=.2\textwidth
}
\begin{lstlisting}[caption={Weitere allgemeine Firewallregeln},label=lst:allgemeineRegeln]
# Erlaube ICMP Befehle
-A INPUT -p icmp -j ACCEPT
-A OUTPUT -p icmp -j ACCEPT

# Erlaube SSH Verbindung
-A INPUT -p tcp --dport 22 -j ACCEPT

# Alle Verbindungen von innen nach aussen zulassen
-A FORWARD -i eth0 -o eth1 -m state --state NEW -j ACCEPT

# Erlaube nur bereits aufgebaute 
# Verbindungen von aussen nach innen
-A FORWARD -m state --state RELATED,ESTABLISHED -j ACCEPT
\end{lstlisting}
\vspace{\baselineskip}
Die in Auflistung \ref{lst:allgemeineRegeln} dargestellten Regeln sind allgemeine Regeln für die Firewall, diese erlauben das empfangen von Pinganfragen, den Fernzugriff mittels SSH, alle neuen Verbindungen von innen nach außen und alle bereits aufgebauten Verbindungen beziehungsweiße alle Verbindungen die einen Bezug auf eine andere Verbindung besitzen. In der nachfolgenden Tabelle \ref*{tab:iptablesOptionen} werden die hier verwendeten Optionen nochmals näher beschrieben.
\begin{table}[h]
	\centering
\begin{tabular}{|p{2cm}|p{14cm}|}
	\hline 
	Optionen & Beschreibung \\ 
	\hline 
	-p & Gibt das Protokoll an welches verwendet werden soll hier icmp(Ping), tcp \\ 
	\hline 
	-dport & Gibt den Destinationport an, auf den zugegriffen werden soll hier 22 \\ 
	\hline
	eth0,eth1 & bezieht sich auf das verwendete Netzwerkinterface \\ 
	\hline 
	-m state & die entreffenden Pakete sollen auf den Status überprüft werden \\ 
	\hline 
	--state & Status der eintreffenden Pakete, hier NEW, RELATED,ESTABLISHED \\ 
	\hline 
\end{tabular} 
\caption{IPTABLES Optionen und Beschreibung} \label{tab:iptablesOptionen}
\end{table}
Nach den allgemeine Regeln müssen nun die VPN spezifischen Regeln definiert werden. Dazu muss man die auf den VPN Ports eintreffenden und ausgehenden Pakete betrachten. Diese Ports sind entweder 1194 oder 1195. In der Auflistung \ref{lst:VPNRegeln} sind VPN Regeln zusehen. Eine Besonderheit bei VPN ist das zum einen alle verbinungen die über das Netzwerkinterface tun0 eintreffenden Pakete ohne Überprüfung an das Inteface eth0 übergeben werden. Des weiteren wird noch die Tabelle NAT benötigt, was das ''*nat'' angibt. In dieser Tabelle gibt es die Chain POSTROUTING, über diese kann nachträglich der Verkehrsheader eines Paketes verändert werden. MASQUERADE hat dabei die Funktion das wenn ein Paket versendet wird die Source-IP-Adresse so verändert wird das nur noch die IP-Adresse des Firewallservers ersichtlich ist. Dies hat den Grund das von außen nicht ersichtlich wird, was sich für IP-Adressen hinter der Firewall befinden und es so Angreifern erschwert wird diese anzugreifen.
\lstset{
	basicstyle=\footnotesize, frame=tb,
	xleftmargin=.2\textwidth, xrightmargin=.2\textwidth
}
\begin{lstlisting}[caption={Weitere VPN Firewallregeln},label=lst:VPNRegeln]
# Erlaube alle Verbindungen auf den VPN Ports
-A INPUT -i eth1 -p udp --dport 1194 -m state --state NEW -j ACCEPT
-A INPUT -i eth1 -p udp --dport 1195 -m state --state NEW -j ACCEPT
-A INPUT -m state --state RELATED,ESTABLISHED -j ACCEPT
-A OUTPUT -m state --state RELATED,ESTABLISHED -j ACCEPT
-A INPUT -i tun0 -j ACCEPT

#Forwarding fuer die VPN Verbindungen
-A FORWARD -i tun0 -o eth0 -m state --state NEW -j ACCEPT
-A FORWARD -i eth0 -o tun0 -m state --state NEW -j ACCEPT
-A FORWARD -i tun0 -o eth0 -m state --state RELATED,ESTABLISHED -j ACCEPT
-A FORWARD -i eth0 -o tun0 -m state --state RELATED,ESTABLISHED -j ACCEPT
COMMIT

#Verwenden der Tabelle NAT
*nat

#Loeschen der vorhandenen Regeln
-F

-A POSTROUTING -o eth1 -j MASQUERADE
COMMIT
\end{lstlisting}
\vspace{\baselineskip}
Nach dem erstellen des Skriptes, muss dieses nun eingespielt werden. Dazu wird der Befehl ''iptables-restore'' verwendet. Dieser wird exemplarisch in der Auflistung \ref{lst:fwrestorea} veranschaulicht, dabei liegt das Skript im Ordner ''etc''.
\newline
\lstset{
	basicstyle=\footnotesize, frame=tb,
	xleftmargin=.2\textwidth, xrightmargin=.2\textwidth
}
\begin{lstlisting}[caption={Einspielen des Firewallskriptes der Firma a},label=lst:fwrestorea]
iptable-restore < etc/iptables.firewall-a.rules
\end{lstlisting}
\vspace{\baselineskip}
Nach dem das Skript eingespielt wurde, muss dieses noch so konfiguriert werden, dass es nach einem Serverneustart automatisch neu eingespielt wird. Dazu könne zwei Arten verwendet werden. Zum einen mit Hilfe des ''if-pre-up.d'' Directories. Dabei wird eine neue Datei angelegt mit dem Namen ''iptables'' angelegt. In der Auflistung \ref{lst:neustart} wird der Inhalt, der neu angelegten Datei gezeigt. Nun muss die Datei noch Ausführbar gemacht werden, dies geschieht mit Hilfe des Befehls ''chmod +x''
\newline
\lstset{
	basicstyle=\footnotesize, frame=tb,
	xleftmargin=.2\textwidth, xrightmargin=.2\textwidth
}
\begin{lstlisting}[caption={Automatisches Laden des Iptablesskriptes bei Serverneustart},label=lst:neustart]
#!/bin/bash
/sbin/iptasbles-restore < /etc/iptables.fwirewall-a.rules
\end{lstlisting}
\vspace{\baselineskip}
Die zweite Möglichkeit des automatischen Neuladens der Firewall ist mit Hilfe des Tools ''iptables-peristent''. Dieses Tool muss mit ''aptitute'' installiert werden. Nach dem das Tool installiert wurde, können mit dem Befehl ''iptables-save > /etc/iptables/rules.v4'' die Regeln für jeden Neustart automatisch geladen werden. 
\newline 
Eine weitere Konfiguration die vorgenommen werden muss, ist das Akzeptieren des Forwardings. Dazu muss im Verzeichnis ''/proc'' die Datei ''/proc/sys/net/ipv4/ip\_forward'' die Zeile in der Auflistung \ref{lst:forward} eingefügt werden. Zu Beginn steht in dieser Datei lediglich eine ''0'' diese muss ersetzt werden durch eine ''1'' dies aktiviert das Forwarding.\newline
\lstset{
	basicstyle=\footnotesize, frame=tb,
	xleftmargin=.2\textwidth, xrightmargin=.2\textwidth
}
\begin{lstlisting}[caption={Forwarding aktivieren},label=lst:forward]
/bin/echo "1" > /proc/sys/net/ipv4/ip_forward
\end{lstlisting}
\vspace{\baselineskip}
Wurden all diese Einstellungen vorgenommen, kann begonnen werden mit der Konfiguration der Site-toSite VPN Verbindung, was im nachfolgenden Kapitel beschrieben wird.



\subsection{VPN}
Nach dem die Firewall auf den Servern der Firma A und B eingerichtet wurde, muss nun die VPN Verbindung erstellt werden. Dazu dient der Firewallserver der Firma A als Server und der Firewallserver der Firma B als Client. Diese Einrichtung wird in diesem Kapitel näher beschrieben. \newline
Dafür muss zu Beginn auf beiden Servern das in Kapitel \ref{openvpn} vorgestellte Tool OpenVPN installiert werden. Dies geschieht wie schon im Kapitel \ref{firewallloesung}  erwähnt mit dem Tool ''aptitute''. Nach der Installation müssen wie nachfolgend erklärt für den Server der Firma A und B ein Serverzertifikat erstellt werden.

\subsubsection{Serverzertifikate}
Wie schon in Kapitel \ref{PKI} erklärt wird für die  Einrichtung eines VPN's Zertifikate benötigt. Diese werden verwendet um eine Verbindungsverschlüsselung zu realisieren. In diesem Kapitelteil wird anhand der Firma A erklärt, wie ein solches Zertifikat in der Laborumgebung erstellt wird, um ein Zertifikat für Firma B zu erstellen, ist der gleiche Prozess notwendig. \newline
Für das Erstellen eines Zertifikates ist das Tool OpenSSL notwendig. Über OpenSSL wird ein öffentlicher und privater Schlüssel erzeugt. Nach dem diese erstellt wurden, muss von der Zertifizierungsstelle die Konfigurationsdatei geladen werden. Mit Hilfe dieser Konfigurationsdatei des privaten Schlüssels und des Tools OpenSSL wird nun eine CSR (Certificate Signing Request) Datei erzeugt. Nach dem diese Datei erzeugt wurde, muss diese zur Zertifizierungsstelle hochgeladen werden. Nach dem Upload der Datei (Zertifikat), muss diese nun Signiert werden. Dies wir im Labor mit einem Shellskript realisiert. Nach dem Signieren kann nun das fertige Zertifikat geladen und auf dem Server 
abgelegt werden. In der Auflistung \ref{lst:openssl}  werden die einzelnen Befehle zur Erstellung der Schlüssel und der CSR Datei gezeigt und die einzelnen Befehle in der Tabelle \ref{tab:openSSLOptionen} näher beschrieben.\newline
\lstset{
	basicstyle=\footnotesize, frame=tb,
	xleftmargin=.2\textwidth, xrightmargin=.2\textwidth
}
\begin{lstlisting}[caption={Erzeugen eines privaten Schlüssels mit OpenSSL},label=lst:openssl]
#Privaten Schluessel erzeugen
openssl genrsa -out firma-A.key 2048

#Request erstellen - CSR-Datei
openssl req -new -key firma-A.key -config req.cnf -reqexts v3_req_srv -out CSR-firma-A.csr
\end{lstlisting}
\vspace{\baselineskip}
\begin{table}[h]
	\centering
\begin{tabular}{|l|l|}
	\hline 
	Befehle & Beschreibung \\ 
	\hline 
	openssl & Tool zur Erstellung von privaten Schlüsseln \\ 
	\hline 
	genrsa & Befehl zum erstellen des Schlüssels mittels RSA \\ 
	\hline 
	-out & Ausgabe des Schlüssels in angegebene Datei \\ 
	\hline 
	firma-A.key & Name der Schlüsseldatei \\ 
	\hline 
	2048 & Schlüssellänge in Bits \\ 
	\hline 
	req -new & Erzeugen eines neuen Requests \\ 
	\hline 
	-config & laden der Konfigurationsdatei req.cnf \\ 
	\hline 
	-reqexts & Erweiterung des X.509 Formates \\ 
	\hline 
\end{tabular} 
\caption{Befehle zur Erstellung eines Privaten Schlüssels mittels OpenSSL} \label{tab:openSSLOptionen}
\end{table}

\subsubsection{Server Firma A}
Nach dem die Zertifikate wie im vorherigen Kapitel erstellt wurden, kann nun mit dem Konfigurieren des VPN-Servers begonnen werden. Dazu wird zuerst mittels des Root-Users das Verzeichnis ''/etc/opnevpn'' erstellt. Nach dem dieses erstellt wurde werden zwei Unterverzeichnisse ''/etc/openvpn/certs'' und ''/etc/openvpn/keys'' angelegt. In das Unterverzeichnis ''certs'' werden das Serverzertifikat ''fw2-firma-A.crt'' und die Zertifikatschain (von der Zertifizierungstelle) kopiert. In das Unterverzeichnis ''keys'' wird der private Schlüssel verschoben. Da der private Schlüssel geheimgehalten werden soll, muss die Berechtigung für das Unterverzeichnis noch geändert werden. Dies wird mit dem Befehl ''chmod -R 600 /etc/openvpn/keys/'' bewerkstelligt. Der nächste Schritt ist das erstellen des Diffie-Hellman Schlüssels. Dieser Schlüssel wird später benötigt um später die Verbindung symmetrisch zu verschlüsseln. Der dafür notwendige Befehl wird in der Auflistung \ref{lst:DH} gezeigt.\newline
\lstset{
	basicstyle=\footnotesize, frame=tb,
	xleftmargin=.2\textwidth, xrightmargin=.2\textwidth
}
\begin{lstlisting}[caption={Erzeugen des Diffie-Hellman Schlüssels},label=lst:DH]
openssl dhparam -out dh2048.pem 2048
\end{lstlisting}
\vspace{\baselineskip}



\subsubsection{Client Firma B}










