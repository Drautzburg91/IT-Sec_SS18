% !TEX encoding = UTF-8 Unicode
\section{Lösungsskizze}
In diesem Kapitel wird sich mit dem Thema befasst, wie sich eine Side-to-Side VPN Verbindung, die die Firmen A und B wie in Kapitel \ref{laborumgebung} beschrieben verbindet.
Dazu waren zu Beginn mehrere Vorüberlegungen zu tätigen. Dies sind zum einen, wie wird das innere Netz der Firmen A und B geschützt, dann welcher VPN Dienst kann für diesen Zweck verwendet werden und welche weiteren Strategien braucht es bei der Umsetzung der Security Policies der Firmen. 
\subsection{Firewall}
Wenn eine Firewall aufgebaut werden soll, muss sich zu beginn überlegt werden, welche allgemeine Strategie mit der Firewall gefahren werden soll. Das heißt sollen alle Verbindungen Standardmäßig freigegeben werden und nur explizit nicht erlaubt Verbindungen geblockt werden (Default-Allow-Strategy), oder sollen alle Verbindungen blockiert werden und nur die die explizit erlaubt wurden, geöffnet werden (Default-Deny-Strategy). Im Standardfall wird in Unternehmen, die Strategie Default-Deny verwendet, weshalb sich auch im Versuchsaufbau diese Strategie gewählt wurde. Eine Firewall wird mit Hilfe des Userspace-Programmes IPTABLES (siehe Kapitel \ref{iptables}) unter Linux realisiert. 